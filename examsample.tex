\documentclass[12pt,twoside]{article}

%\usepackage{hyperref}
\usepackage[questions,exam,writeon,questiontable]{uoftexam}

%----------------------------------------------------------------
% Exam Info

\InstitutionName{UNIVERSITY OF TORONTO}
\FacultyName{Department of Computer and Science}
\DepartmentName{}

\Date{2013}
\CourseTitle{CSC373H3}
\CourseCode{CSC373H3}
\CourseSemester{Fall 2013}
\CourseSemesterCode{F} % F for Fall, S for Spring
\CourseSection{}
\CourseInstructor{Instructor: Kaveh Ghasemloo}

\Title{Examinations}
\Date{December 2013}

\ExamDuration{3 hours}
\ExamAids{No Aids Allowed.}


\ExamInstructions{
\vspace{-1cm}
\begin{itemize}
\item [] 
\begin{center}
{\bf Exam Instructions}
\end{center}

\item 
{\bf Check that your exam book has \pageref{page:last} pages}
(including this cover page and 
\TotalScratchPages{}~blank pages at the end). 
The last \TotalScratchPages{}~pages are for rough work only, 
{\it they will {\bf not} be marked}.
Please bring any discrepancy to the attention of an invigilator.

\item 
There are \QuestionCount{}~questions worth a total of \TotalScore{}~points.
Answer all questions on the question booklet. 
You need to get at least 40\% in this exam to pass the course.

\item 
In questions 1\textendash 3, 
if you don't know the answer
you can leave the question blank {\it and} 
write \textquotedblleft{\sc I~don't~know.}\textquotedblright {}
to receive 20\% of the points of the question.

\item 
Please read all questions right now and 
ask any clarification questions you have during the first 60 minutes 
to minimize distractions to other students.

\item [] 
\begin{center}
{\bf Course Specific Notes}
\end{center}

\item 
Unless stated otherwise, 
you can use standard data structures and algorithms 
discussed in CSC263 and CSC373
by simply stating their standard name 
(e.g. min-heap, merge-sort, Dijkstra)
without describing their implementation
or proving their properties.
If you modify a data structure or an algorithm from class, 
you must describe the modification and its effects.

\end{itemize}
}


%\generalnotes{Notations}{
%}


\begin{document}
%----------------------------------------------------------------
% The top and marking pages
%----------------------------------------------------------------
\toppage

%----------------------------------------------------------------
% Questions Begin
%----------------------------------------------------------------
\begin{enumerate}

%----------------------------------------------------------------
%\newexampage
%\newsection{Mathematical Background}
%----------------------------------------------------------------
\newexampage
\newsection{A. 1ST SECTION}
%----------------------------------------------------------------
% Question 1
%----------------------------------------------------------------
\newquestionsn{}{QUESTION 1}{

This is a question with two parts:

\begin{enumerate}
\newquestionpart{10}{
This is the 1st part!
}

\newpadding{1}

\newquestionpart{5}{
This is the 2nd part!
}

\newpadding{1}

\end{enumerate}
}


\newsolution{
This is the solution:

\begin{enumerate}
\newsolutionpart{
Solution for the 1st part!
}

\newsolutionpart{
Solution for the 2nd part!
}
\end{enumerate}
}

%----------------------------------------------------------------
% Question 2
%----------------------------------------------------------------
\newexampage
\newquestionsn{20}{QUESTION 2}{

This is another question.
}

\newanswerpage

%----------------------------------------------------------------
\newexampage
\newsection{B. 2ND SECTION}
%----------------------------------------------------------------
% Question 3
%----------------------------------------------------------------
\newquestionsn{15}{QUESTION 3}{

This is yet another question.
}


%----------------------------------------------------------------
% Questions End
%----------------------------------------------------------------
\end{enumerate}


\newscratchpage{2}

%----------------------------------------------------------------
% The bottom page
%----------------------------------------------------------------
\bottompage

\end{document}
